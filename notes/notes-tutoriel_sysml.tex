% Created 2020-10-31 sam 14:41
% Intended LaTeX compiler: pdflatex
\documentclass[11pt]{article}
\usepackage[utf8]{inputenc}
\usepackage[T1]{fontenc}
\usepackage{graphicx}
\usepackage{grffile}
\usepackage{longtable}
\usepackage{wrapfig}
\usepackage{rotating}
\usepackage[normalem]{ulem}
\usepackage{amsmath}
\usepackage{textcomp}
\usepackage{amssymb}
\usepackage{capt-of}
\usepackage{hyperref}
\author{Dalker}
\date{\today}
\title{SysML - notes de DK}
\hypersetup{
 pdfauthor={Dalker},
 pdftitle={SysML - notes de DK},
 pdfkeywords={},
 pdfsubject={},
 pdfcreator={Emacs 27.1 (Org mode 9.3)}, 
 pdflang={English}}
\begin{document}

\maketitle
\tableofcontents

\section{sources:}
\label{sec:orge729a81}
\begin{itemize}
\item \url{https://sysml.org/}
\item \url{https://www.omg.org/spec/SysML/} et notamment un pdf "tutoriel" disponible dans omg.org
\end{itemize}
\section{"taxonomie" de différents types de diagrammes}
\label{sec:orgd52020e}
-> cf. png extrait du pdf "turoriel" d'omg.org:
 "behaviour" (activity, sequence, state machine, use case)
 "requirement"
 "structure" (block definition, internal block [+ parametric], package)
\section{Frame}
\label{sec:orgb345512}
tout diagramme doit avoir:
\begin{itemize}
\item un \textbf{contour} (frame)
\item une \textbf{en-tête} en haut à gauche indiquant le type de diagramme, le type
d'élément modélisé et le nom de l'élément modélisé + éventuel nom
personalisé
\item une \textbf{description} éventuelle avec no. de version, état d'achèvement
(completion status), évtl desc et refs
\end{itemize}
\section{Structural diagrams}
\label{sec:org47fb0cd}
\subsection{package diagram}
\label{sec:org001df6e}
organise le modèle; devrait utiliser la relation \textbf{import} pour des
sous-parties
\subsection{block}
\label{sec:org557980b}
\begin{itemize}
\item décrit un élément d'un système
\item le bloc se subdivise en "compartiments" (rectangles emplilés) pour différents
caractéristiques
\item based on UML Class + flow ports + value properties
\end{itemize}
\subsection{property of a block}
\label{sec:orgc99eb05}
\begin{itemize}
\item part: block inside a composite block (which "owns" it): black diamond in
bdd, plain rectangle in ibd
\item reference: block not owned by aggregating block: white diamond in bdd,
dashed rectangle in ibd
\item value: number with units or probability distribution
\end{itemize}
\subsection{block diagrams (2 types)}
\label{sec:org60edbc3}
\begin{itemize}
\item block definition diagram: relationship between blocks (composition,
assoiciation, specialization) with usual uml triangle/diamonds
\item internal block diagram: internal structure of one block (properties and
connectors) with ports (std or flow)
\end{itemize}
\subsection{ports}
\label{sec:org9cb1e46}
\begin{itemize}
\item std (UML): 
closed circle = provided interface (provides the operation)
open circle = required interface (calls the operation)
when connected, simple line
\item flow (SysML extension):
little square with arrow pointing out or in
when connected, black arrow in connecting line
\end{itemize}
\subsection{parametrics}
\label{sec:org54fe164}
def of constaint: block with title, equation and def of parameters
param diagram: connectors between equations and parameters are white squares
inside the plain rectangles (params) or rounded rectangles (equations)
\section{Behavioural diagrams}
\label{sec:org254067d}
\subsection{Activity diagram}
\label{sec:org864b37d}
\begin{itemize}
\item transformation from input to output via sequence of actions
\item routing flows: fork / join ("and", black rectangle), decision / merge ("or",
white diamond)
\item types of flow: 1) object / data, 2) control
\begin{enumerate}
\item enters/exits action via plain arrows through attached white squares on
the sides or above/below
\item enters/exits action via dashed arrows, vertically, directly to the
action's rounded rectangle
\end{enumerate}
\item unit of flow: "token", consumeed and produced by actionsj
\item an action can invoke (be zooomed into) another activity
\item inputs and outputs can be required or optional ("\label{org57d4cfc}"), and
optionals can also be part of a continuous stream ("\{stream\}")
\item action starts when token present in all control inputs and all required
inputs
\item activity's in/out showed by rectangles crossing frame boundary
\item special actions: accept event ("bookmark shape") and send signal ("arrow
shape")
\item "swimlanes" can be \label{org149fcef}d to separate behaviour
\end{itemize}
\subsection{Sequence diagram}
\label{sec:org6664622}
for message based behaviour: flow of control, interaction between parts
\subsection{State machines}
\label{sec:org75c6104}
show life cycle of a block
support behaviour based on events: change / time / signal
\subsection{Use cases}
\label{sec:org1081437}
functinality in terms of usage/goals by actors; no change to UML
heavy use of \label{orgb8fe610} and \label{orge750f93}
\section{Allocations and requirements}
\label{sec:org1566c9f}
\subsection{allocations}
\label{sec:org5baac11}
relations between models:
\begin{itemize}
\item behavioural (function to component)
\item structural (logical to physical)
\item software to hardware
\item etc\ldots{}
\end{itemize}
Activities may be explicitily allocated to structure via swim lanes (activity partitions)
\label{org3669108} relation may be used instead (with dashed arrow or on
super-rectangle), and reverse with \textbf{\textbf{allocatedFrom}} label or property
\subsection{requirements}
\label{sec:org8af952a}
relationships: DeriveReqt, Satisfy, Verify, Refine, Trace, Copy
client --> supplier relations: if supplier changes, client must adapt
(i.e. typically client implements something needed by supplier)
\section{Connecting Model Elements}
\label{sec:org282e496}
\begin{itemize}
\item allocation: \textbf{behaviour} (possibly swimlines) is/are allocated to \textbf{structure} blocks
\item satisfaction: \textbf{structure} satisfies \textbf{requirements}
\item parametrics: structure binds them, requirements verify them
\end{itemize}
\section{Example: distiller problem -> see p 75 and forward (illustrates all types of diagram)}
\label{sec:orgbc5796d}
\section{System Modeling Activities}
\label{sec:org2465301}
\begin{itemize}
\item Analyze needs: causal analysis, mission use cases / scenarii, enterprise model
\item Define System Requirements: system use cases / scenarii, elaborated context
\item Define Logical Architecture: Logical decomposition, scenarii and subsystems
\item Synthesize Allocated Architecture: Node diagram, HW/SW/Data arch, deployment
\end{itemize}
\begin{itemize}
\item also:
\begin{itemize}
\item Optimize / Evaluate Alternatives (parametric diagram, etc.)
\item Manage requirements (req't diagram and/or tables)
\item Support Validation and Verification (test cases)
\end{itemize}
\end{itemize}
\section{2ème Exemple: système de sécurité de maison, p. 102 et suivantes}
\label{sec:org957d4f8}
\end{document}